User data is collected and processed in many applications.
From a privacy perspective, this is bad for the users --- it is a privacy risk,
that data might leak or be used for something unintended by the user.
In many cases we actually do not need the exact user data, e.g.\ if we want to 
estimate user behaviour or do statistical computations over all users.
One tool that is available to use is differential privacy.
Here we review the definitions of what it is and look at a variety of uses.

In particular, after this session you should be able to
\begin{itemize}
  \item \emph{understand} the uses of differential privacy and to which kind of 
    problem it can be applied.
  \item \emph{analyse} different solutions in the research literature based on 
    differential privacy and \emph{suggest} a design for a solution to a given 
    problem using the relevant findings.
\end{itemize}

The basic theory of differential privacy is covered by Chapters 1--2 (with more
in Chapter 3) in 
\citetitle{DifferentialPrivacyBook}~\cite{DifferentialPrivacyBook}.
We will furthermore look at some interesting applications of differential 
privacy in:
\begin{itemize}
  \item \citetitle{ChallengingDiffPriv}~\cite{ChallengingDiffPriv},
  \item \citetitle{BLIP-1}~\cite{BLIP-1},
  \item \citetitle{BLIP-2}~\cite{BLIP-2},
  \item \citetitle{RAPPOR}~\cite{RAPPOR}, and
  \item \citetitle{RAPPOR-unknown}~\cite{RAPPOR-unknown}.
\end{itemize}
