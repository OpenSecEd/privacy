\title{Differential Privacy}
\subtitle{A short introduction}
\author[D. Bosk <dbosk@kth.se>]{Daniel Bosk}
\institute[KTH CSC]{%
  School of Computer Science and Communication\\
  KTH Royal Institute of Technology\\
  \email{dbosk@kth.se}
}
\date{12th February 2016}

\maketitle

\mode*

\begin{abstract}
  User data is collected and processed in many applications.
From a privacy perspective, this is bad for the users --- it is a privacy risk,
that data might leak or be used for something unintended by the user.
In many cases we actually do not need the exact user data, e.g.\ if we want to 
estimate user behaviour or do statistical computations over all users.
One tool that is available to use is differential privacy.
Here we review the definitions of what it is and look at a variety of uses.

In particular, after this session you should be able to
\begin{itemize}
  \item \emph{understand} the uses of differential privacy and to which kind of 
    problem it can be applied.
  \item \emph{analyse} different solutions in the research literature based on 
    differential privacy and \emph{suggest} a design for a solution to a given 
    problem using the relevant findings.
\end{itemize}

The basic theory of differential privacy is covered by Chapters 1--2 (with more
in Chapter 3) in 
\citetitle{DifferentialPrivacyBook}~\cite{DifferentialPrivacyBook}.
We will furthermore look at some interesting applications of differential 
privacy in:
\begin{itemize}
  \item \citetitle{ChallengingDiffPriv}~\cite{ChallengingDiffPriv},
  \item \citetitle{BLIP-1}~\cite{BLIP-1},
  \item \citetitle{BLIP-2}~\cite{BLIP-2},
  \item \citetitle{RAPPOR}~\cite{RAPPOR}, and
  \item \citetitle{RAPPOR-unknown}~\cite{RAPPOR-unknown}.
\end{itemize}

\end{abstract}


\section{Differential Privacy}

\subsection{Background}

\begin{frame}
  \begin{block}{Dalenius, 1977}
    Nothing about an individual should be learnable from the database that 
    cannot be learned without access to the database.
  \end{block}

  \pause{}

  \begin{block}{Dwork, 2006\footfullcite{DifferentialPrivacy}}
    \begin{itemize}
      \item Dwork proves this is impossible.
      \item Actually you can learn things about individuals not in the 
        database.
    \end{itemize}
  \end{block}
\end{frame}

\begin{frame}
  \begin{block}{Origin: Statistical databases}
    \begin{itemize}
      \item Privacy-preserving data analysis
      \item Statistical disclosure control
      \item Inference control
      \item Privacy-preserving data-mining
      \item Private data analysis
    \end{itemize}
  \end{block}

  \pause{}

  \begin{block}{Purpose}
    Protect \emph{any} one entry when publishing statistics about a database.
  \end{block}
\end{frame}

\begin{frame}
  \begin{block}{Non-interactive}
    \begin{itemize}
      \item Compute and publish some statistics.
      \item Database not used further.
      \item What can we infer from the released statistics?
    \end{itemize}
  \end{block}

  \pause{}

  \begin{block}{Interactive}
    \begin{itemize}
      \item Interactively answer statistical queries about the database.
      \item Queries and/or answers may be modified by the privacy mechanism.
    \end{itemize}
  \end{block}
\end{frame}

\subsection{Definition}

\begin{frame}
  \begin{block}{The idea}
    \begin{itemize}
      \item We want to protect \emph{any} one entry in this database.

        \pause{}

      \item Thus we want the result to be similar with and without this entry.

        \pause{}

      \item Thus we add random noise to achieve this.
    \end{itemize}
  \end{block}
\end{frame}

\begin{frame}
  \begin{definition}[Differential 
    privacy\footfullcite{DifferentialPrivacyBook}]
    \begin{itemize}
      \item Let \(\M\) be a randomized function.
      \item Let \(D, D'\in \ker(\M)\) be databases differing on at most one 
        element.
      \item Let \(S\subseteq \im(\M)\).
      \item \(\M\) is \(\epsilon\)-differentially private if and only if
        \[\Pr[\M[D]\in S] \leq e^\epsilon \Pr[\M[D']\in S]\]
      \item \color{red} Probability taken over coin-tosses of \(\M\).
      \end{itemize}
  \end{definition}
\end{frame}

\begin{frame}
  \begin{remark}
    \begin{itemize}
      \item \enquote{Probability taken over coin-tosses of \(\M\).}
      \item As opposed to the \enquote{probability over the differences of \(D, 
            D'\)}.
    \end{itemize}
  \end{remark}
\end{frame}

\begin{frame}
  \begin{block}{Note}
    This protects a row even if the adversary knows every other row!
  \end{block}
\end{frame}

\begin{frame}
  \begin{definition}[Sensitivity]
    \begin{itemize}
      \item Let \(f\) be the true function, i.e.\ not \(\M\).
      \item \(D, D'\) as before.
      \item The sensitivity of \(f\) is \[
          \Delta f = \max_{D, D'} || f(D) - f(D') ||_{1}.
        \]
    \end{itemize}
  \end{definition}
\end{frame}

\begin{frame}
  \begin{theorem}
    \begin{itemize}
      \item \(f\) as before.
      \item \(\M_f\) is the mechanism adding noise to \(f\).

        \pause

      \item If \(\M_f\) adds noise with \(\Lap(\Delta f/\epsilon)\),
        then \(\M\) provides \(\epsilon\)-differential privacy.
    \end{itemize}
  \end{theorem}

  \pause

  \begin{remark}
    The needed noise depends only on the sensitivity of \(f\) and \(\epsilon\).
  \end{remark}
\end{frame}

\begin{frame}
  \begin{example}[Histograms]
    \begin{itemize}
      \item Histogram query with \(k\) cells.

        \pause{}

      \item Viewed as \(k\) counting queries.

        \pause{}

      \item Adding or removing a database entry can affect at most one of \(k\)
        cells.

        \pause{}

      \item Thus the histogram function has sensitivity \(1\).
    \end{itemize}
  \end{example}
\end{frame}


\section{Uses}

\subsection{Matching profiles}

\begin{frame}
  \begin{block}{The idea}
    \begin{itemize}
      \item Differential privacy is designed for statistics.
      \item {\color{red} We must be able to add noise.}

        \pause

      \item Computing a similarity score can be based on statistics:
        \begin{itemize}
          \item Is item \(X\) in the database? (\(Count(X)\))
          \item That's a statistical question.
        \end{itemize}
    \end{itemize}
  \end{block}
\end{frame}

\begin{frame}
  \begin{example}[Standard use]
    \begin{itemize}
      \item Have a statistical database.
      \item Each user's data corresponds to one row.

        \pause{}

      \item We protect individual users, by protecting the rows.
    \end{itemize}
  \end{example}

  \pause

  \begin{example}[BLIP\footfullcite{BLIP-2}]
    \begin{itemize}
      \item Let a user profile be the database.
        \begin{itemize}
          \item Each row is a \enquote{like}.
          \item Or a photo or a comment.
        \end{itemize}

        \pause{}

      \item Protect the individual rows.
        \begin{itemize}
          \item Protect the contents of the profile.
        \end{itemize}
    \end{itemize}
  \end{example}
\end{frame}

\begin{frame}
  \begin{block}{BLoom-and-flIP, BLIP}
    \begin{itemize}
      \item Hash the entries of the profile into a Bloom filter (\(L\)-bit 
        string).

        \pause

      \item Generate a random \(L\)-bit string.
      \item XOR them together --- adds noise.
    \end{itemize}
  \end{block}

  \pause

  \begin{block}{Results\footfullcite{BLIP-2}}
    \begin{itemize}
      \item BLIP is \(\epsilon\)-differentially private.
      \item So you can do reconstruction \enquote{within \(\epsilon\)}.
    \end{itemize}
  \end{block}
\end{frame}


\begin{frame}
  \begin{question}[Isn't it a guarantee?]
    \begin{itemize}
      \item You are guaranteed \(\epsilon\)-differential privacy.
      \item So there is something the attacker may learn.
      \item But how much is that?
      \item How do we choose \(\epsilon\)?
    \end{itemize}
  \end{question}
\end{frame}

\begin{frame}
  \begin{block}{Practical attacks\footfullcite{ChallengingDiffPriv}}
    \begin{itemize}
      \item \citet{ChallengingDiffPriv} constructs two practical attacks.
      \item These reconstruct the profiles in BLIP (within \(\epsilon\)).

        \pause

      \item This is one step closer to an informed choice for \(\epsilon\).
    \end{itemize}
  \end{block}
\end{frame}

\subsection{Analytics: collecting user data}

\begin{frame}
  \begin{example}[RAPPOR\footfullcite{RAPPOR}]
    \begin{itemize}
      \item Local differential privacy: yields correct population statistics.
      \item Estimate client-side distribution of strings.
      \item Used in Chrome to track distribution of configurations.
    \end{itemize}
  \end{example}
\end{frame}

\subsection{Machine learning}

\begin{frame}
  \begin{remark}
    \begin{itemize}
      \item Machine learning is \enquote{just statistics}.
    \end{itemize}
  \end{remark}

  \pause

  \begin{example}
    \begin{itemize}
      \item Can ensure that a model is differentially private.
      \item Differential privacy guarantees protection from over-fitting.
    \end{itemize}
  \end{example}
\end{frame}

\begin{frame}
  \begin{example}[Federated learning]
    \begin{itemize}
      \item Want a model trained on databases \(D_1, \dotsc, D_n\).
      \item The owner of \(D_i\) don't want to leak its database.
      \item Can create output \(O_i\) which yields differential privacy for 
        \(D_i\).
      \item E.g.~GBoard
    \end{itemize}
  \end{example}
\end{frame}


%%% REFERENCES %%%

\begin{frame}[allowframebreaks]
  \printbibliography
\end{frame}
