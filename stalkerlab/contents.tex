\title{Lab: Privacy is dead}
\author{Daniel Bosk}
\affil{%
  Department of Information and Communication Systems\\
  Mid Sweden University, Sundsvall
}

\maketitle
\tableofcontents


\section{Introduction}%
\label{sec:introduction}
Since our world becomes more connected and information becomes more accessible, 
easier to publish, copy and save in different databases, the need for 
protection increases.
There are several examples of how ``innocently looking'' data are published 
which yield that disastrous information can be derived.

One example is the Norwegian royal family.
One of the children in the family published photographs from their vacation to 
Instagram.
These photographs happened to contain geotagging information, which per se is 
not that dangerous.
The problem was that the rate and time of publication gave the information 
about their position in almost real-time~\cite{Roberts2012wia}.

The same problem can arise for ordinary people, i.e.\ not-high-security 
targets.
For instance, a proof of concept service, WeKnowYourHouse.com, parsed Twitter 
feeds for data and derived where people lives and if they were at 
home~\cite{Brading2012tpl}.

The final example is by Samy Kamkar.
In his talk ``How I Met Your Girlfriend''~\cite{Kamkar2010him} he shows how to 
use the easy accessibility of personal information and technical 
vulnerabilities to set up a date with Anna Faris.


\section{Scope and Aim}%
\label{sec:aim}
The aim of this assignment is to illustrate unintentional spreading of personal 
information and how many small pieces of information can be combined to derive 
new information and the possibilites for being misused.

As such, after the completion of this assignment you should be able to:
\begin{itemize}
	% $Id$
% Author:	Daniel Bosk <daniel.bosk@miun.se>
\item Reflect on the security aspects of different types of social behaviour.

\end{itemize}


\section{Theory}%
\label{sec:theory}
The theoretical support for this assignment will be given during the time of 
the course.
Since the assignment starts at the beginning of the course and ends in the end, 
you will successively get more and more tools at your disposal.


\section{Assignment}%
\label{sec:assignment}
The assignment is to collect as much information as possible about the 
principal lecturer of the course.
In the end of the course all students in the class will present their findings 
in a seminar, thus together mapping all collected information.

The information collected may vary from hobbies, food preferences, phone 
numbers, addresses for summer houses, etc.\ to working habits, e.g.\ office 
hours, email response times, and so on.
After this assignment you should know more about the teacher than the teacher 
himself.

However, there are some requirements:
\begin{enumerate}
  \item The target of the information gathering \emph{must not} suspect 
    anything, e.g.\ by excessively asking around -- then colleagues might bring 
    it to the target's attention.
    If the target becomes aware of this, you will get a ``boooh'' from the 
    entire class during the presentation -- this is of course very awkward.

  \item You must be able to provide a source of the information (leak).
    E.g.\ ``the target mentioned during the lecture on 1/4 2013 that he does 
    skiing'' or ``a colleague mentioned that he usually comes in around 
    9 o'clock''.

  \item You must also provide an estimate of how trustworthy the information 
    is.
    The reason for this is that you might find information which is plainly 
    wrong or just misinterpreted.
    There is also the risk that some information presented will contradict 
    other information, then you need the estimate of trustworthyness to resolve 
    the conflict.
\end{enumerate}


\section{Examination}%
\label{sec:exam}
All collected information will be presented orally, from all participants, 
during the presentation.
You can find the time for this in the course schedule, it will be in the end of 
the course.


\printbibliography{}
