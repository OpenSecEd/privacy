% $Id$
\documentclass[a4paper,nocourse]{miunasgn}
\usepackage[utf8]{inputenc}
\usepackage[english,swedish]{babel}
\usepackage[hyphens]{url}
\usepackage{hyperref}
\usepackage{prettyref,varioref}
\usepackage{natbib}
\usepackage{listings}
\usepackage[today,nofancy]{svninfo}
\usepackage[natbib,varioref,prettyref,listings]{miunmisc}

\svnInfo $Id$

%\printanswers

\courseid{IG036G}
\course{Informationssäkerhet}
\assignmenttype{Laboration}
\title{Privacy is dead}
\author{Daniel Bosk\footnote{%
	Detta verk är tillgängliggjort under licensen Creative Commons 
	Erkännande-DelaLika 2.5 Sverige (CC BY-SA 2.5 SE).
	För att se en sammanfattning och kopia av licenstexten besök URL 
	\url{http://creativecommons.org/licenses/by-sa/2.5/se/}.
}}
\date{\svnId}

\begin{document}
\maketitle
\thispagestyle{foot}
\tableofcontents


\section{Introduktion}
\label{sec:introduction}
\noindent
Då vår värld blir alltmer uppkopplad och information blir enklare att 
publicera, kopiera och spara i olika databaser ökar även behovet att skydda 
informationen.
Ett exempel som illustrerar enkelheten att publicera till synes ofarlig 
information på ett sätt som ger skadliga effekter är en incident som drabbade 
det norska kungahuset.
Där publicerades den norska kungafamiljens geografiska position nästintill 
i realtid -- information som av säkerhetsskäl normalt hålls hemlig -- via 
geotaggade semesterbilder publicerade hos Instagram \cite{Roberts2012wia}.

Samma säkerhetslucka kan ha stora konsekvenser även för vanliga människor, 
trots att deras position normalt inte är sekretessbelagd, då det kan användas 
exempelvis för att veta när huset är mest tillgängligt för inbrott.
Ett proof-of-concept för detta är WeKnowYourHouse.com \cite{Brading2012tpl} som 
söker igenom Twitterposter efter geotaggar och formuleringar som ''äntligen är 
jag hemma''.

Sedan har vi Samy Kamkars föreläsning ''How I met your girlfriend'' 
\cite{Kamkar2010him} där han nyttjar tillgängligheten av personlig information 
och svagheter i systemen som skyddar denna för att boka in en dejt med Anna 
Faris.


\section{Syfte}
\label{sec:aim}
\noindent
Syftet med denna laboration är:
\begin{itemize}
	% $Id$
% Author:	Daniel Bosk <daniel.bosk@miun.se>
\item Reflect on the security aspects of different types of social behaviour.

\end{itemize}


\section{Teori}
\label{sec:theory}
\noindent
Det teoretiska stödet för denna uppgift kommer att ges alltefter kursen går.
Uppgiften kommer att börja redan vid introduktionsföreläsningen och fortgå till 
ett redovisningstillfälle i slutet av kursen då all kursens teori är 
genomgången.


\section{Genomförande}
\label{sec:work}
\noindent
Du ska under kursens gång samla på dig information om kursansvarig lärare, 
målet, för att i slutet av kursen göra en fullständig kartläggning av denne.
Informationen kan variera från fritidssysselsättningar; matpreferenser; 
telefonnummer; adresser till bostäder, sommarstugor och dylikt; till 
arbetsvanor och vilken buss personen tar hem från jobbet.
Du ska efter genomförandet veta mer om vederbörande än vederbörande själv.

De enda krav som finns är följande:
\begin{enumerate}
  \item Målet inte får misstänka någonting.
    Om målet blir varse att du försökt hitta information får du ett ''bu'' av 
    alla närvarande vid redovisningstillfället -- vilket naturligtvis vore 
    ytterst pinsamt.
  \item Du måste kunna ange en (möjligen) anonym\footnote{%
      Det finns inget krav på att ni ger källan anonymitet, är det exempelvis 
      en före detta student som uttalat sig negativt kan ni ge dem källskydd.
      Men det finns ingen egentlig anledning att göra detta mot exempelvis 
      kollegor eller Facebook.
    } källa för informationsläckaget.
    Exempelvis ''målet yttrade under föreläsningen den 1/4 2013 att denne åkte 
    skidor på fritiden'' eller ''en kollega berättade att målet brukar komma in 
    vid niotiden''.
  \item Du måste ge en uppskattning på tillförlitligheten till informationen.
    Detta är först och främst för att det naturligtvis finns risken att ni 
    hittar falsk information.
    Det finns även en risk att vid redovisningstillfället presenteras 
    information som säger emot annan information.
    Ni behöver då tillförlitligheten för att kunna nå konsensus över vilken 
    information som mest sannolikt är riktig.
\end{enumerate}


\section{Examination}
\label{sec:Examination}
\noindent
All insamlad information redovisas muntligen och alla deltagares insamlade 
information sammanställs vid ett redovisningstillfälle i slutet av kursen.


\bibliography{literature}
\end{document}
